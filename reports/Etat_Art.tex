\documentclass{article}

\usepackage[utf8]{inputenc}
\usepackage[T1]{fontenc}
\usepackage{geometry}
\geometry{a4paper}

\title{État de l'Art}
\author{Munch Alexis, Vaillant Erwan, Yang Michaël}
\date{\today}

\begin{document}

\maketitle

\section{Introduction}

Dans le domaine en constante évolution de la robotique, une nouvelle frontière a
été franchie avec l'émergence de la matière programmable. Cette technologie,
composée d'un grand nombre d'unités élémentaires, ou robots modulaires, offre la
possibilité de créer des structures complexes à partir de composants
relativement simples. Chaque unité est équipée de capteurs et d'effecteurs qui
lui permettent de se déplacer et de s'accrocher à d'autres éléments de la
structure, ainsi que d'une capacité de calcul qui leur permettent d'exécuter des
programmes.

Cependant, un défi majeur se pose : comment programmer ces robots modulaires
pour qu'ils réalisent une structure avec une forme prédéfinie ? Deux approches
sont envisagées. La première consiste à définir un programme unique qui serait
dupliqué sur toutes les unités, leur permettant de planifier leurs mouvements
jusqu'à leur position finale. La seconde approche serait de permettre aux unités
d'apprendre par essais et erreurs et de se configurer elles-mêmes, l'utilisateur
guidant le travail en fixant une forme et en laissant les robots se programmer
eux-mêmes.

Ce projet vise à explorer ces deux voies et à proposer des algorithmes
d'apprentissage et d'adaptation qui permettront aux robots élémentaires de
décider de leurs actions. Le travail sera principalement réalisé en simulation
sur le logiciel VisibleSim, spécialement conçu pour programmer ces robots en
C++.

Cependant, malgré les avancées dans ce domaine, il existe encore des lacunes et
des problèmes non résolus. Par exemple, la question de savoir comment garantir
l'efficacité et la fiabilité des programmes de robots modulaires reste ouverte.
De plus, le processus d'apprentissage par essais et erreurs peut être long et
complexe, et la manière d'optimiser ce processus n'est pas encore claire. Enfin,
la question de savoir comment l'utilisateur peut guider efficacement le travail
reste aussi un point important. Ces défis constituent le cœur de ce projet et
seront abordés tout au long de l'étude.

\section{Definition, Domaine de recherche}

\section{Historique}

\section{Méthodologie}

Dans le domaine de la matière programmable, plusieurs méthodes sont couramment
utilisées pour programmer les robots modulaires. En plus des deux méthodes
mentionnées précédemment, à savoir, la modélisation prédictive
et l'apprentissage par essais et erreurs - il existe d'autres approches qui
méritent d'être explorées.

1. Programmation comportementale : Cette méthode consiste à programmer chaque
robot avec un ensemble de comportements simples qui, lorsqu'ils sont combinés,
permettent au système de réaliser des tâches complexes. Les comportements
peuvent être déclenchés par des stimuli spécifiques, permettant aux robots de
réagir de manière autonome à leur environnement.

2. Optimisation par essaim : Inspirée par le comportement des colonies
d'insectes, cette méthode utilise des algorithmes d'optimisation pour guider le
mouvement et l'assemblage des robots. Chaque robot agit comme un agent d'un
essaim, travaillant en collaboration avec les autres pour atteindre un objectif
commun.

3. Apprentissage par renforcement : Cette méthode utilise le machine learning pour permettre aux robots d'apprendre de leurs erreurs et d'améliorer
leurs performances au fil du temps. Les robots reçoivent des récompenses ou des
punitions en fonction de leurs actions, ce qui les guide vers les comportements
qui maximisent leur récompense globale.

Chacune de ces méthodes a ses avantages et ses inconvénients. La programmation
comportementale est relativement simple à mettre en œuvre et peut être très
efficace pour certaines tâches, mais elle peut ne pas être suffisamment flexible
pour s'adapter à des environnements comportant plusieurs robots. L'optimisation par
essaim peut permettre une coordination efficace entre les robots, mais elle peut
être plus difficile à mettre en œuvre. Enfin, le machine learning peut permettre une grande flexibilité et une adaptation continue,
mais il nécessite beaucoup de données et de temps pour l'apprentissage.

Il est important de noter que ces méthodes ne sont pas mutuellement exclusives
et peuvent souvent être combinées de manière bénéfique.

\section{Outils et Technologies}

\section{Synthèse et conclusion}

\end{document}